\begin{abstract}
This study investigates the application of Bayesian analysis in flood frequency hydrology, employing an adaptive Differential Evolution Markov Chain with Snooker Updater (DE-MCzS) algorithm for parameter estimation. The results were verified against BestFit, a state-of-the-art hydrological software developed by the U.S. Army Corps of Engineers Risk Management Center. Two case studies are presented: (1) a stationary flood frequency analysis of the Harricana River in Canada, which assesses the robustness of the Bayesian framework and its advantages over classical statistical methods, and (2) a non-stationary analysis of the O.C. Fisher Dam in the United States, examining temporal variability in flood frequency parameters. Through verification, we demonstrate the advantages of the Bayesian approach in terms of integrating uncertainty, accommodating temporal dynamics, and incorporating expert knowledge. By addressing challenges such as data limitations and evolving climatic conditions, it offers a robust framework for enhancing decision-making in hydrological risk management and infrastructure design.

\FloatBarrier
\end{abstract}