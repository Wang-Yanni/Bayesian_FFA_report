\section{Conclusion}
This study demonstrated the application of a Bayesian computational framework in flood frequency hydrology. We utilized an adaptive DE-MCzS algorithm for the MCMC process. The approach was validated through two case studies using publicly available datasets, with results verified against BestFit, a leading hydrological software developed by USACE. 

The stationary analysis of the Harricana River dataset evaluated the efficiency and robustness of the adaptive DE-MCzS algorithm while also demonstrating the advantages over MLE. The computed posterior mode, posterior predictive distributions, and credible intervals aligned exceptionally well with BestFit when the same noise term was applied. The non-stationary analysis of the O.C. Fisher Dam dataset further emphasized the Bayesian approach’s capability in modeling temporal dynamics as a time-variant property of the location parameter. Both stationary and non-stationary models (NSFFA with linear and exponential trends) produced results closely aligned with BestFit. Sensitivity analysis on the TI selection revealed the significant influence of temporal parameters on flood frequency curves. 

By incorporating prior knowledge and explicitly quantifying uncertainty, the Bayesian approach proved to be a powerful tool for managing flood risks, especially in the context of data scarcity, increasing uncertainties, and the evolving impacts of climate change. This integration of probabilistic reasoning with hydrological modeling enhances decision-making processes, offering robust solutions to the complex challenges faced in flood risk assessment and management.