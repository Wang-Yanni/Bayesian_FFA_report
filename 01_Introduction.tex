\section{Introduction}
Flood Frequency Analysis (FFA) is a fundamental aspect of hydrological studies, providing essential insights for the design of infrastructure, risk assessment, and water resource management. Accurate prediction of extreme flood events and their probabilities is critical for mitigating flood impacts, especially in regions vulnerable to climate variability and extreme weather events. Traditionally, statistical distributions for FFA have been estimated using Maximum Likelihood Estimation (MLE), which provides deterministic estimates based on available data. While effective, MLE does not inherently account for the uncertainties and variability often present in hydrological datasets, which limits its ability to fully represent the probabilistic nature of flood risks.

The Bayesian approach offers a robust alternative, providing a probabilistic framework that integrates uncertainty and prior knowledge into the analysis. By leveraging prior distributions and computing posterior probabilities, Bayesian methods enable hydrologists to quantify uncertainty in parameter estimates, yielding a more comprehensive understanding of flood risks. This capability is particularly valuable in hydrology, where data limitations and temporal variability in processes are common challenges. Additionally, Bayesian methods support the incorporation of expert judgment and allow for model updating as new data becomes available.

This study aims to provide a verification of the Bayesian computational framework implemented in BestFit, a software tool developed by the U.S. Army Corps of Engineers Risk Management Center (USACE RMC) to enhance and expedite advanced flood hazard assessments \citep{Smith_2020}. It has been widely used within the Flood Risk Management, Planning, and Dam and Levee Safety communities. Its latest release (version 2.0) in October 2024, introduces advanced capabilities for non-stationary FFA (NSFFA), enabling hydrologists to model temporal changes in flood frequency parameters \citep{Smith_2024}.

In the Bayesian estimation analysis, we employ the adaptive Differential Evolution Markov Chain with Snooker Updater (DE-MCzS) algorithm for the Markov Chain Monte Carlo (MCMC) process, as implemented in BestFit. Two case studies are presented to validate the computational framework: (1) a stationary analysis of the Harricana River in Canada, which compares the performance of the Bayesian approach against classical MLE, evaluates the DE-MCzS algorithm, and verifies posterior distributions with BestFit; and (2) both stationary and non-stationary FFA applied to the O.C. Fisher Dam in the United States, focusing on the verification of NSFFA with BestFit, comparison between stationary and non-stationary models, and examining the influence of temporal variability on flood frequency curves.

\FloatBarrier
